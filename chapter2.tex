\chapter{Experimental Detector Design}

\section{Relativistic Heavy Ion Collider}

\section{sPHENIX Detector}

The super Pioneering High Energy Nuclear Interaction eXperiment (sPHENIX) detector is located at RHIC and employs an array of central tracking, calorimeter and global event characterization subsystems in order to make measurements of the heavy ion and proton-proton collisions that occur from the RHIC beams. 
sPHENIX is the first general purpose detector to be built on a particle collider in the last decade and has been built with specific goals of measuring jets, heavy flavor and quarkonia at RHIC energies for direct comparison with measurements in the last decade from the LHC experiments.
The full array of sPHENIX physics program goals can be found in Fig. \ref{fig:sphenix_physics_program}.
\begin{figure}
    \centering
    \includegraphics[width=\linewidth]{Figures/sPHENIXPhysicsProgram.png}
    \caption{Diagram of full array of sPHENIX physics program goals including measurements in jet structure, heavy flavor, quarkonia, bulk heavy ion physics and proton spin physics.}
    \label{fig:sphenix_physics_program}
\end{figure}

sPHENIX began data taking in 2023 with a commissioning dataset of Au+Au collisions at $\sqrt{s_{NN}} = 200$ GeV. 
In 2024, sPHENIX took its premire physics dataset of p+p collisions at $\sqrt{s} = 200$ GeV sampling over 100 pb$^{-1}$ allowing for precision jet measurements in p+p collisions at RHIC energies.
sPHENIX also collected an additional dataset of Au+Au collisions at $\sqrt{s_{NN}} = 200$ GeV in 2024 intended for the commissioning of the tracking detectors with the calorimeter system and global detectors in full physics datataking mode. 
This Au+Au dataset was very useful in validating the calorimeter and jet reconstruction, calibration and preformance and provided confidence in the sPHENIX data collection and reconstruction prior to the sPHENIX high statistics flagship Au+Au dataset collected in 2025 sampling 7 nb$^{-1}$. 
The results shown in this thesis are from the 2024 p+p and Au+Au datasets. 

sPHENIX has three main subsystem groups: a 4-part tracking system with a MAPS-based vertex detector (MVTX), an intermediate silicon tracker (INTT), the time projection chamber (TPC) and the TPC Outer Tracker (TPOT), 
a central barrel calorimeter system with three calorimeter layers of electromagnetic and hadronic calorimetry (EMCal, IHCal and OHCal), 
and global detectors including the Minimum Bias Detector (MBD), Zero Degree Calorimeters (ZDCs) and sPHENIX Event Plane Detector (sEPD).
A rendering of the sPHENIX detector highlighting these subsystems with labels can be found in Fig. \ref{fig:sphenix_schematic}.

\begin{figure}
    \centering
    \includegraphics[width=0.9\linewidth]{Figures/sphenix_rendering.png}
    \caption{sPHENIX detector with subsystems labeled.}
    \label{fig:sphenix_schematic}
\end{figure}

\subsection{sPHENIX Tracking System}
The sPHENIX tracking system comprises four subdetectors: a MAPS-based vertex detector (MVTX), an intermediate silicon tracker (INTT), the time projection chamber (TPC) and the TPC Outer Tracker (TPOT).
These tracking subdetectors can provide a combination of particle momentum measurements, particle species identification, and secondary vertex reconstruction for use in physics measurements espeically related to heavy flavor physics and jet substructure. 

The MVTX is the innermost tracking subsystem, at a distance of 2-5 cm from the beam pipe, and is a 3 layer silicon barrel design with full azimuthal coverage ($0 < \phi < 2 \pi$) and pseudorapidity coverage of $|\eta| < 1$.
Each layer of silicon barrel is composed of a set of staves, for a total of 48 staves, which each contain a hybrid integrated circuit with a flexible printed circuit connected to nine ALICE Pixel Detector (ALPIDE) pixel chip sensors, for a total of 432 ALPIDE sensors \cite{MVTX,MVTX_proceeding}.
The ALPIDE sensors are Monolithic Active Pixel Sensors (MAPS) that use Complementary Metal Oxide Semiconductor (CMOS) technology to build the pixel directly on the read out chip. 
The sensor consists of a silicon die of size 15 mm x 30 mm with an active p-doped epitaxial layer where charge is freed by an incoming particle and a set of charge collection diodes (pixels) with position resolution of 30 $\mu$m where the charge is then collected. 
Following collection, the sensors also amplify and digitize signals from each pixel and zero suppression is applied before read out. 
The MVTX is essential in sPHENIX's ability to make heavy flavor measurements \cite{MVTX2024heavyflavorphysicssphenix}. 
By utilizing its high precision position resolution and streaming readout capabilities, the MVTX can reconstruct a large sample of decay vertices from heavy flavor hadrons. 
\begin{figure}
    \centering
    \begin{subfigure}{\linewidth}
        \centering
        \includegraphics[width=0.8\linewidth]{Figures/MVTX_stave_and_half_layer.png}
        \caption{MVTX stave and half-barrel layer}
    \end{subfigure}
    \hfill
    \begin{subfigure}{\linewidth}
        \centering
        \includegraphics[width=0.8\linewidth]{Figures/MVTX_detector_barrel.png}
        \caption{MVTX detector barrel}
    \end{subfigure}
    \caption{Top: Schematic of an sPHENIX MVTX stave (left) and half barrel layer made up of staves and service section (right). Bottom: Cross-section of MVTX silicon barrel with its 3 layers highlighted in different colors (Red, Green, and Blue) and sPHENIX beam pipe included to highlight the MVTX location relative to the beam pipe. Reproduced from \cite{MVTX}.}
    \label{fig:sphenix_mvtx_stave}
\end{figure}

The INTT is a two-layer silicon barrel subsystem located between the MVTX and TPC with the first layer at a radial distance of 7.5 cm and second layer at 10 cm and has full azimuthal coverage, which is ensured by staggering the silicon sensors of the two layers in $\phi$. 
The INTT barrels are composed of silicon ladders containing silicon strip sensors, readout chips, high-density interconnect cables and a carbon fiber stave that serves as a support structure \cite{INTT_beamtest2025}. 
At the silicon sensor, electron-hole pairs are created when charge particles pass through the detector.
This signal is then digitized, zero suppression is applied and the data is read out via the high-density interconnect cables to readout cards.
INTT hits can be used to bridge the MVTX and TPC tracks to improve tracking reconstruction. 
However, the INTT is also essential in associating reconstructed tracks with data from the sPHENIX calorimeter system and global detectors which are readout in triggered mode.
This is done by using the precise timing resolution of the INTT hits, therefore reconstructed tracks can be associated with RHIC bunch crossings by using the timing of their INTT hits \cite{INTT_beamtest2025}. 

\begin{figure}
    \centering
    \includegraphics[width=0.8\linewidth]{Figures/intt_ladder.png}
    \caption{Schematic of sPHENIX INTT silicon ladder component. Highlighted on silicon ladder are silicon sensors, readout chips and high-density interconnect (HDI) cables. Reproduced from \cite{INTT_beamtest2025}.}
    \label{fig:sphenix_intt}
\end{figure}

The TPC is the main sPHENIX tracking subdetector, spanning in radial distance from 20 to 80 cm with full azimuthal coverage ($0 < \phi < 2 \pi$) and pseudorapidity coverage of $|\eta| < 1.1$.
The sPHENIX TPC is a cylindrical double-sided TPC design with a central membrane at the center of the detector and two readout endcaps on the interior of either side of the detector's span in z.
In each half of the TPC, an electric field is formed by setting the central membrane electrode to a high voltage and the readout endcaps to ground \cite{sPHENIX_TDR,TPC_overview}.
This electric field is constrained in the radial direction by field cages on the inner and outer cylindrical surfaces of the TPC.
The TPC volume is also filled with a gas mixture of Argon, CF4 and Isobutane in a ratio of 75:20:5.
When charged particles from the collision pass through the TPC gas volume, they ionize the gas creating electrons and ions, which act as signal for use in determining particle trajectories through the TPC. 
These freed electronics then drift towards the readout endcaps via the electric field between the central membrane and the readout endcaps. 

On each end of the TPC there is an amplification region where Gaseous Electron Multipliers (GEM) modules are used to amplify the signal from ionized electrons within the TPC volume.
Each GEM module contains a stack of four Kapton + Copper foils with a varying voltage across each foil creating a high electric field within the holes of the foils. 
As electrons pass through the full GEM stack they undergo avalanche amplification, where each foil contributes a small amount of the overall electron amplification.
The electric fields between the GEM foils are also able to minimize the amount of avalanche induced positive ions that drift back into the TPC volume thereby reducing ion back flow and allowing the TPC to operate continuous readout. 
Finally the amplified electron signals are read out using SAMPA chips developed by ALICE each which contain 32 channels. 
The TPC has a total of 159,744 readout channels read out through a set of 6, 8 and 12 Front End Electronics (FEE) corresponding to the inner (R1), middle (R2) and outer (R3) radial regions of the TPC endcaps.

\begin{figure}
    \centering
    \includegraphics[width=0.8\linewidth]{Figures/tpc_diagram.png}
    \caption{Left: Cross-section schematic of sPHENIX TPC showing central membrane (CM), inner- and outer-field cages (I-FC, O-FC) and end plates (EP). 
    Middle: Segementation of TPC endcap into 72 GEM modules. 
    Right: Schematic of TPC GEM modules highlighting the four foil layers with varying electric fields and pad plane for readout of amplified signal. Reproduced from \cite{sPHENIX_TDR}.}
    \label{fig:sphenix_tpc}
\end{figure}

The TPOT 

\subsection{sPHENIX Calorimeter System}

sPHENIX has three central barrel calorimeter layers, an electromagnetic calorimeter (EMCal), an inner hadronic calorimeter (IHCal) and an outer hadronic calorimeter (OHCal). 
\textcolor{red}{Add in sentence about calorimeter necessary for jet physics goals and about that the caloriemter system measures energy and is used as a trigger}
These detectors are concentric and sandwich the sPHENIX magnet with the EMCal located closest to the interaction point and the OHCal located further from the interaction point within the calorimeter system. 
Further, these detectors have full azimuthal coverage ($0 < \phi < 2 \pi$) and pseudorapidity coverage of $|\eta| < 1.1$. 

The sPHENIX EMCal is designed to precisely measure photons, electrons and positrons via electromagnetic showers.
To acheive this goal, the EMCal has a depth of 18 radiation lengths and it's electromagnetic shower resolution was measured in beam tests to be 2.8\% $\oplus$ 15.5\%/$\sqrt{E}$ for electrons \cite{emcal_beamtest,emcal_hcal_beamtest}.
The EMCal is also nearly one nuclear interaction length in depth and therefore sees a substantial amount of energy in the ion collisions from hadronic showers as well. 
The sPHENIX EMCal is a compact sampling calorimeter; it is located just outside of the TPC/TPOT with an inner radius of 90 cm and the EMCal and IHCal combined are constrained to fit within the repurposed Babar magnet. 
The EMCal uses a tungsten powder absorber, which has a small Moliere radius and large particle stopping power. 
This allows the EMCal to have a large number of radiation lengths and precision position resolution while still being compact. 
Scintillating fibers embedded within the tungsten powder collect and transmit light to a lightguide on the front of each EMCal block (size 2x2 towers). 
Following collection by the lightguide, the signals are proccessed by Silicon Photomultiplers (SiPMs), which convert the light to electric current, and passed through a pre-amplifier.
The analog voltage signals are then passed to the Analog-Digital Converters as the start of the sPHENIX calorimeter data aquistition system. 
The EMCal is composed of 64 sectors, 2 sectors in $\eta$ and 32 sectors in $\phi$, with each sector containing 384 towers for a total of 24,576 EMCal towers of tower size $\Delta \eta \times \Delta \phi = 0.024 \times 0.024$.
Each sector has six interface boards used to communicate with the detector for setting and monitoring the detector state, including information about the SiPM temperatures, bias voltage and offsets supplied to the SiPMs, leakage current, and detector gain. 
These slow control communication boards are also used for detector commissioning tasks such as turning on and pulsing LED and test pulses for calibration runs.

The sPHENIX HCal system is the first central barrel hadronic calorimeter on a detector at RHIC and allows for full jet reconstruction with both electromagnetic and hadronic calorimetry at RHIC energies for the first time. 
sPHENIX is designed to measure hadronic showers with the full calorimeter system totaling a depth of 5 nuclear interaction lengths and having a hadron resolution response in beam tests of 13.5\% $\oplus$ 64.9\%/$\sqrt{E}$ \cite{emcal_hcal_beamtest}. 
The HCal system is split into two layers, the IHCal and OHCal, separated by the Babar superconducting magnet repurposed for sPHENIX.
The IHCal is located between the EMCal and the magnet with an inner radius of ??? cm and the OHCal is located outside of the magnet acting as the magnet flux return with an inner radius of ??? cm and an outer radius of ??? cm.
The IHCal (OHCal) are sampling calorimeters composed of aluninum (steel) aborber plates and 4 (5) scintillating tiles set at an angle offset to the transverse direction to reduce the amount of collision particles that do not interact with the active volumes of these calorimeters. 
Wavelength shifting fibers are embedded within the scintllating tiles which collect and transmit the produced light to SiPMs.
The SiPMs process this light into voltage signals which is then amplified and digitized in the same manner as for the EMCal.
Both the IHCal and OHCal contain 32 sectors split into North and South half-sector sides and each sector contains 48 towers for a total of 1536 towers of tower size  $\Delta \eta \times \Delta \phi = 0.01 \times 0.01$.
Each half sector has one interface and slow control communication board and these are also used to set and monitor the detector state including SiPM temperatures, bias voltages and offsets, leakage current and detector gain.

\subsection{sPHENIX Global Detectors}
sPHENIX contains several global detectors used to provide mimumium bias event triggering and total event characterization information such as the vertex of the collision event and centrality of the event.
These detectors include the Minimum Bias Detector (MBD), Zero Degree Calorimeters (ZDCs) and sPHENIX Event Plane Detector (sEPD). 

The MBD is located at forward rapidity, $3.51 < |\eta| < 4.61$, on both the north and south side of the interaction point, close to the beam pipe, and is repurposed from the PHENIX Beam Beam Counter (BBC) with updated trigger and readout electronics, where it was located in a different position $3.0 < |\eta| < 3.9$ along the beam line \cite{phenix_bbc1,phenix_bbc2}.
Each side of the MBD contains a set of 64 1-in diameter mesh-dynode photomultiplier tubes (PMT) and the front of these PMTs are connected with a 3 cm quartz Cherenkov radiator to detect charged particles via Cherenkov radiation.
During both PHENIX and sPHENIX running, the MBD has been located close to the beam pipe and is subject to high levels of radiation during data taking.
In it's original design, quartz was selected for the radiator material and PMTs for signal amplification due to their radiation hardness and over the data taking lifetime of the MBD, the subsystem has not seen any signification degradation due to radiation damage.
The MBD is used for online triggering of hadronic interactions and offline to provide a selection of minimum-bias (MB) events and centrality determination in heavy-ion collisions. 
It is also used for the offline reconstruction of the collision z-vertex position using the arrival time difference between the first charged particles detected by the PMTs on the north and south side of the interaction point. 

The ZDCs~\cite{zdc} are located on both sides of the interaction point at very forward rapidity, 18 m from the interaction point, and are also repurposed from the PHENIX detector. 
They are sampling hadronic calorimeters composed of three modules made of tungsten alloy and PMMA optical fiber with a total of 5.1 nuclear interation lengths (149 radiation lengths).
Ultizing their very forward location, the ZDCs are able to detect beam fragment neutrons of the collision without additional backgrounds from produced particles and secondaries from the collision.
A schematic showing how the sPHENIX ZDC detector measures these beam fragment neutrons while not being sensitive to beam fragment charged particles can be found in Fig. \ref{fig:sphenix_zdc}.
\begin{figure}
    \centering
    \includegraphics[width=0.8\linewidth]{Figures/zdc_profile.png}
    \caption{Schematic of sPHENIX ZDC detector shown from (a) collision region view and (b) "beam's eye" view to show the ZDC location past the RHIC DX dipole magnets and between the beam pipes. Also shown is the trajectory of beam fragments with neutrons measured in the region of the ZDC and protons deflected. Reproduced from \cite{zdc}.}
    \label{fig:sphenix_zdc}
\end{figure}
The total energy of the neutrons are measured in the ZDCs via the transmission of Cherekov radiation from the neutron's hadronic showers by PMMA fibers to PMTs.
Each side of the ZDC is calibrated by requiring that the peak in the measured energy distribution which corresponds to the energy deposition of a single neutron is set at the nominal beam energy value of 100 GeV. 
Additionally, specifically in sPHENIX, criteria on signals in both sides of the ZDC are included in the minimum bias event selection to selarate hadronic interactions from beam backgrounds like beam-gas interactions.

The sEPD 

\subsection{sPHENIX Data Acquisition and Readout}
sPHENIX is designed with a state of the art data aquisition system and computing resources to collect data at very high rates which is necessary to achieve it's data-intensive physics goals within it's short three year running period.
The sPHENIX data aquisition system is designed to readout data from triggered calorimeter events at a rate of 15 kHz in both p+p and Au+Au collisions and read out tracking detectors in full streaming mode collecting all of the supplied collision data.
With the high collision rate (>> 15 kHz) supplied by RHIC, the sPHENIX trigger system is used to make decisions using information from the calorimeter and global detectors on whether to read out a particular collision event.
This trigger system uses calorimeter information to identify and save likely high $p_{T}$ photon and jet events and uses the global detectors to identify and save minimum bias hadronic collision events.

The triggered calorimeter and global detector data is read out by going through in five steps: 
first the front end modules (FEM) which digitize the analog signals from the SiPMs and PMTs in the detectors, 
second the data collection modules (DCMs) which combine data from multiple FEMs and act as very short term buffers, 
the data is then passed through the sub event buffer (SEB) computers used to store the data temporarily, and then is transferred to bufferboxes, before finally being written to disk storage.
The tracking data is read out from the front end electronics (FEEs) through the ATLAS "FELIX" card which is connected directly to a standard computer which serves as a buffer similar to the triggered SEBs.
The tracking data is also transferred to bufferboxes and written to disk storage in the same manner as the triggered data.

The Really Cool Data Acquisition (RCDAQ) system is used to control and coordinate the entire sPHENIX data aquisition and readout process. 
The RCDAQ coordinates the start and stop of all of sPHENIX's indepedent read out data streams and tramission of data of variable types with fixed packet sizes through the different stages of buffer computers.
This is necessary to ensure that all subsystems are sychronized in their read out and is especially important since in sPHENIX all subsystem compnents have indpendent data stream read out and there is no online event building.
Via the transmission of data packets with fixed sizes, the RCDAQ is also able develop a controlled and steady flow for the data through the read out system.