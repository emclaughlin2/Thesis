\chapter{Introduction and Background}

\section{Footnotes: Two ways of adding to your text}

\footnote{By using footnote command and writing your note in the curly brackets}. Or it is possible to mark the location of a foot note with footnote mark command\footnotemark \, then you can write the footnote in its own line for ease of reading. 

\footnotetext{You then use footnotetext command and then write you note in as if you are using regular footnote command as we did previously.}

\section{Other section of first chapter}

% This is a table
%If you are having issues with \midline use \hline insteadn and remove booktabs package from thesis.tex

\begin{table}
\caption{This is an example Table.}
\begin{center}
\begin{tabular}{ccc}
x & f(x) & g(x) \\
%\hline
\midrule
1 & 6 & 4  \\
2 & 6 & 3  \\
3 & 6 & 2  \\
4 & 6 & 2  \\
\label{Table in Chapter 1}
\end{tabular}
\end{center}
\end{table}
