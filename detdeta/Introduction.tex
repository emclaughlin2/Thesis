Transverse energy has been long proposed as a golden probe to study the characteristics of the quark-gluon plasma (QGP). The transverse energy per unit pseudorapidity (\dEtdeta) probes the energy carried by the medium along the longitudinal rapidity, providing essential information related to the initial geometry and subsequent hydrodynamic evolution of the QGP. Previous measurements from PHENIX \cite{PhysRevC.93.024901} and STAR \cite{PhysRevC.70.054907} at RHIC have demonstrated that such a strongly interacting medium is produced in \auau collisions at a nucleon-nucleon center-of-mass energy of \snn = 200~GeV. It has been found that the created system behaves as an almost perfect fluid. Similar characteristics have been noted in various experiments at the LHC with Pb+Pb collisions \cite{Adam_2016,PhysRevLett.109.152303} at higher energies.

This note details the analysis of the \dEtdeta measurements with the calorimeter system of the sPHENIX detector, comprising the Electromagnetic Calorimeter (EMCal) and the Inner and Outer Hadronic Calorimeter (HCal). The observable \dEtdeta is defined as the transverse energy per unit pseudorapidity measured using the calorimetry system. The energy deposited an individual tower of the calorimeters is used as the unit to construct the transverse energy. The Minimum Bias Detector (MBD) and Zero Degree Calorimeter (ZDC) are employed for the event selection and centrality determination. 