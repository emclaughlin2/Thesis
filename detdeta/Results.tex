Final \dEtdeta results using EMCal, HCal and Full Calorimeter data are presented as a function of $\eta$ for all centrality bin measurements in Fig.~\ref{fig:detdeta_full_cent_range}. The \dEtdeta values for all three measurements with the sPHENIX EMCal, HCal and full calorimeter have a strong dependence on centrality, increasing towards more central \auau collisions, whereas no significant dependence on $\eta$ is seen. Additionally, these measurements are consistent with one another within uncertainties. Further, for all calorimeter measurements, \dEtdeta at positive $\eta$ and negative $\eta$ are compatible within uncertainties; symmetry with respect to $\eta$ is highlighted in Fig.~\ref{fig:flipped_detdeta_full_cent_range}, where the EMCal, HCal and Full Calorimeter results are overlaid by these measurements flipped across $\eta$ = 0. A complete set of the final \dEtdeta results including all calorimeter configurations for all centrality bins studied in this analysis are included in Appendix~\ref{sec:full_detdeta_plots}.

\begin{figure}
    \centering
    \includegraphics[width=\linewidth]{Figures/dETdeta_measurements/emcal_hcal_calo_detdeta_separate.pdf}
    \caption{Results of \dEtdeta over measurement range $-1.1<\eta<1.1$ for EMCal (left), HCal (middle) and Full Calorimeter (right) data. Measurements over centrality range from 0-5\% to 50-60\% presented for all calorimeter configuration measurements.}
    \label{fig:detdeta_full_cent_range}
\end{figure}

\begin{figure}[h]
    \centering
    \includegraphics[width=\linewidth]{Figures/dETdeta_measurements/flipped_emcal_hcal_full_calo_all_cent_log_separate_1.pdf}
    \caption{Results of \dEtdeta over measurement range $-1.1<\eta<1.1$ for EMCal (left), HCal (middle) and Full Calorimeter (right) data shown in red. Measurements over centrality range from 0-5\% to 50-60\% presented for all calorimeter configuration measurements. \dEtdeta measurements flipped about $\eta$ = 0 overlaid in blue to highlight symmetry in the measurements about $\eta$ = 0.}
    \label{fig:flipped_detdeta_full_cent_range}
\end{figure}

Results using all calorimeter configurations of the average $\dEtdeta$ per participant pair, $\Npart$/2, as a function of $\Npart$, are shown in Fig.~\ref{fig:dETdeta_npart_all_calo_layers}. The $\langle \dEtdeta\rangle/(0.5 \Npart)$ values show a gradual increase for more central collisions (i.e. higher \Npart) irrespective of various calorimeter configurations used and good agreement over the full \Npart range is seen for the EMCal, HCal, Full Calorimeter and OHCal-only measurements; the IHCal-only measurement is consistently higher than the other measurements, but sees only 4\% of the collision energy due to the small interation length of the IHCal making it unable to capture the full dynamics of the total collision energy with the same accuracy as the other two calorimeter subsystems which see a much larger fraction of the collision energy. 

\begin{figure}[h]
    \centering
    \includegraphics[width=0.48\linewidth]{Figures/dETdeta_measurements/dETdeta_Npart_vs_Npart_all_calo_layers.png}
    \caption{Mean \dEtdeta over 0.5\Npart as a function of \Npart for EMCal, HCal, IHCal-only, OHCal-only and full calorimeter measurement. Uncertainty bands on measurements correspond to the quadrature sum of the errors on \dEtdeta measurement and the uncertainties on the \Npart calculation.}
    \label{fig:dETdeta_npart_all_calo_layers}
\end{figure}

In the left plot of Fig.~\ref{fig:dETdeta_npart}, the EMCal and HCal \dEtdeta measurements are overlaid to highlight their agreement. This is a particularly encouraging result as the EMCal and HCal see different contributions of the collision energy. Fig~\ref{fig:dETdeta_npart} (left) also includes comparisons to previous STAR and PHENIX results, highlighting the agreement between the three detectors as well as the similar uncertainty size between the sPHENIX EMCal-only measurement and the PHENIX and STAR measurements which also used electromagnetic calorimetry.

Additionally, in the right plot of Fig.~\ref{fig:dETdeta_npart}, sPHENIX Full Calorimeter results are compared to the results from PHENIX \cite{PhysRevC.93.024901} and STAR \cite{PhysRevC.70.054907} and to latest theoretical models. The mean \dEtdeta over 0.5\Npart values are in agreement with previous measurements from PHENIX and STAR across the entire studied \Npart range. Fig.~\ref{fig:dETdeta_npart} also shows comparisons to the truth mean \dEtdeta over 0.5\Npart from HIJING, EPOS and AMPT; across the full \Npart range, AMPT best describes the trend of mean transverse energy across the full \Npart range measured by sPHENIX.

\begin{figure}
    \centering
    \includegraphics[width=0.45\linewidth]{Figures/dETdeta_measurements/emcal_hcal_star_phenix_detdeta_0_5.pdf}
    \includegraphics[width=0.45\linewidth]{Figures/dETdeta_measurements/dETdeta_Npart_vs_Npart_detector_comp_rw_epos.pdf}
    \caption{Left: EMCal and HCal measurements of \dEtdeta as a function of $\eta$ with comparisons to STAR and PHENIX results for most central measurement bin, 0-5\%. \\
    Right: Mean \dEtdeta over 0.5\Npart as a function of \Npart for sPHENIX full calorimeter measurement with comparisons to STAR, PHENIX, and generator-level \dEtdeta from theoretical models, EPOS, HIJING and AMPT. Uncertainty bands on measurements correspond to the quadrature sum of the errors on \dEtdeta measurement and the uncertainties on the \Npart calculation.}
    \label{fig:dETdeta_npart}
\end{figure}