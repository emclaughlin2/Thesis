\subsection{AMPT}

The version of AMPT used in this analysis is the latest build(v1.26t9b-v2) from the \href{https://myweb.ecu.edu/linz/ampt/}{AMPT website}.

The generator setting used for this study is as followed:

\begin{verbatim}
200            ! EFRM (sqrt(S_NN) in GeV if FRAME is CMS)
CMS             ! FRAME
A               ! PROJ
A               ! TARG
197             ! IAP (projectile A number)
79              ! IZP (projectile Z number)
197             ! IAT (target A number)
79              ! IZT (target Z number)
10000		! NEVNT (total number of events)
0.              ! BMIN (mininum impact parameter in fm) 
20.		! BMAX (maximum impact parameter in fm, also see below)
4		! ISOFT (D=4): select Default AMPT or String Melting(see below)
150		! NTMAX: number of timesteps (D=150), see below
0.2		! DT: timestep in fm (hadron cascade time= DT*NTMAX) (D=0.2)
0.55		! PARJ(41): parameter a in Lund symmetric splitting function
0.15    	! PARJ(42): parameter b in Lund symmetric splitting function
1	      	! (D=1,yes;0,no) flag for popcorn mechanism(netbaryon stopping)
1.0	      	! PARJ(5) to control BMBbar vs BBbar in popcorn (D=1.0)
1		! shadowing flag (Default=1,yes; 0,no)
0		! quenching flag (D=0,no; 1,yes)
2.0		! quenching parameter -dE/dx (GeV/fm) in case quenching flag=1
2.0		! p0 cutoff in HIJING for minijet productions (D=2.0)
2.265d0  	! parton screening mass in fm^(-1) (D=2.265d0), see below
0		! IZPC: (D=0 forward-angle parton scatterings; 100,isotropic)
0.33d0		! alpha in parton cascade (D=0.33d0), see parton screening mass
1d6		! dpcoal in GeV
1d6		! drcoal in fm
11		! ihjsed: take HIJING seed from below (D=0)or at runtime(11)
13150909	! random seed for HIJING
8		! random seed for parton cascade
0		! flag for K0s weak decays (D=0,no; 1,yes)
1		! flag for phi decays at end of hadron cascade (D=1,yes; 0,no)
0		! flag for pi0 decays at end of hadron cascade (D=0,no; 1,yes)
1		! optional OSCAR output (D=0,no; 1,yes; 2&3,more parton info)
0		! flag for perturbative deuteron calculation (D=0,no; 1or2,yes)
1		! integer factor for perturbative deuterons(>=1 & <=10000)
1		! choice of cross section assumptions for deuteron reactions
-7.		! Pt in GeV: generate events with >=1 minijet above this value
1000		! maxmiss (D=1000): maximum # of tries to repeat a HIJING event
3		! flag on initial and final state radiation (D=3,both yes; 0,no)
1		! flag on Kt kick (D=1,yes; 0,no)
0		! flag to turn on quark pair embedding (D=0,no; 1,yes)
7., 0.		! Initial Px and Py values (GeV) of the embedded quark (u or d)
0., 0.		! Initial x & y values (fm) of the embedded back-to-back q/qbar
1, 5., 0.       ! nsembd(D=0), psembd (in GeV),tmaxembd (in radian).
0 		! Flag to enable users to modify shadowing (D=0,no; 1,yes)
1.d0		! Factor used to modify nuclear shadowing
0		! Flag for random orientation of reaction plane (D=0,no; 1,yes)
\end{verbatim}


\subsection{EPOS}
EPOS used in this analysis is version 4.0.0 from \href{https://klaus.pages.in2p3.fr/epos4/code/version}{EPOS official website}, our local build is built under gcc 4.8.5 with one modification to the original source code that fixes the conversion of the vertex time unit.

the input setting file used for this study is as followed:

\begin{verbatim}
    !-------------------------------------------------------------
!       AuAu collisions with hydro and hadronic
!-------------------------------------------------------------
application hadron !hadron-hadron, hadron-nucleus, or nucleus-nucleus collision                                                   
set laproj 79  !projectile atomic number
set maproj 197 !projectile mass number
set latarg 79  !target atomic number
set matarg 197 !target mass number
set ecms 200  !sqrt(s)_pp

set istmax 25 
set iranphi 1 
ftime on 

!suppressed decays: 
nodecays 110 20 2130 -2130 2230 -2230 1130 -1130 1330 -1330 2330 -2330 3331 -3331 
9300 -9300 9297 -9297 end

set ninicon 1            !number of initial conditions used for hydro evolution
core full                !core/corona activated 
hydro hlle               !hydro activated 
eos x3ff                 !eos activated
hacas full               !hadronic cascade activated  
set nfull 40              !number of events
set nfreeze 10            !number of freeze out events per hydro event 
set modsho 100           !certain printout every modsho events
set centrality 0         ! 0=min bias 

set ihepmc 1             !write out hepmc file
\end{verbatim}


\subsection{Event Plane afterburner}
Since the EPOS and AMPT implementation used in this analysis doesn't randomize the event plane angle, an \href{https://github.com/sPHENIX-Collaboration/coresoftware/pull/2433}{event plane afterburner} is applied to randomize the reaction plane in the HEPMC record and rotate vertex positions and particle momenta accordingly.

\begin{figure}[!htb]
    \centering
    \includegraphics[width=0.95\linewidth]{Figures/Generator/eventplane.png}
    \caption{Transverse energy as a function of $\phi$ for AMPT before(red) and after(blue) appling the event plane afterburner.}
    \label{fig:enter-label}
\end{figure}